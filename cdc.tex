\documentclass[12pt,a4paper]{article}

% --- Packages de base ---
\usepackage[utf8]{inputenc}
\usepackage[T1]{fontenc}
\usepackage[french]{babel}
\usepackage{graphicx}
\usepackage{xcolor}
\usepackage{hyperref}
\usepackage{geometry}
\usepackage{titlesec}
\usepackage{enumitem}

% --- Configuration des couleurs INSA ---
\definecolor{insa-red}{RGB}{193, 0, 42}
\definecolor{insa-magenta}{RGB}{236, 0, 140}

\geometry{margin=2.5cm}
\hypersetup{
    colorlinks=true,
    linkcolor=insa-red,
    urlcolor=blue,
    pdftitle={Cahier des Charges - App Sportive INSA}
}

% --- Mise en page des titres ---
\titleformat{\section}{\color{insa-red}\normalfont\Large\bfseries}{\thesection}{1em}{}
\titleformat{\subsection}{\color{insa-red!80}\normalfont\large\bfseries}{\thesubsection}{1em}{}

% --- Informations du document ---
\title{
    \vfill
    \Huge \textbf{Cahier des Charges Fonctionnel}\\
    \large Application Événementielle Sportive INSA\\
    \vfill
}
\author{Équipe Projet INSA}
\date{\today}

\begin{document}

\maketitle
\newpage

\tableofcontents
\newpage

\section{Présentation du Projet}
Ce document définit les spécifications techniques et fonctionnelles pour la création d'une application web destinée à un événement sportif inter-INSA regroupant environ \textbf{1000 participants}.

\section{Analyse des Besoins Fonctionnels}

\subsection{Gestion des Utilisateurs et Inscriptions}
\begin{itemize}
    \item \textbf{Authentification :} Accès restreint aux adresses \texttt{@insa-xxx.fr}.
    \item \textbf{Profil :} Choix du sport (liste prédéfinie) et de l'école d'origine.
    \item \textbf{Paiement :} Intégration d'un module de paiement sécurisé (API Stripe/Lydia).
\end{itemize}

\subsection{Informations en Temps Réel}
\begin{itemize}
    \item \textbf{Flux d'actualités :} Page principale avec annonces et notifications "push".
    \item \textbf{Planning :} Calendrier dynamique des épreuves du week-end.
    \item \textbf{GPS :} Cartographie des lieux d'épreuves intégrée.
\end{itemize}

\subsection{Interaction et Scoring}
\begin{itemize}
    \item \textbf{Live Score :</u> Mise à jour des résultats en direct par le staff.
    \item \textbf{Système de Vote :} Module permettant d'élire la meilleure délégation/école durant l'événement.
\end{itemize}

\subsection{Logistique et "Badging"}
\begin{itemize}
    \item \textbf{QR Code Dynamique :} Génération d'un pass unique pour la restauration.
    \item \textbf{Technologie :} Support du NFC pour les terminaux compatibles.
\end{itemize}

\section{Spécifications Techniques}

\subsection{Contraintes de Charge}
L'application doit supporter une charge nominale de \textbf{1000 utilisateurs simultanés}. Une architecture basée sur des \textit{WebSockets} est préconisée pour les scores, avec une mise en cache via \textbf{Redis}.

\subsection{Identité Visuelle}
L'interface doit proposer deux thèmes commutables :
\begin{itemize}
    \item \textbf{Thème Standard :} Rouge (\texttt{\#C1002A})
    \item \textbf{Thème CVL :} Magenta (\texttt{\#EC008C})
\end{itemize}

\section{Interface Administrateur (Staff)}
Le panel d'administration devra permettre :
\begin{itemize}
    \item La gestion des inscriptions et l'export des listes.
    \item La modification des plannings en temps réel.
    \item Le scan des badges pour la gestion des repas.
\end{itemize}

\section{Sécurité}
\begin{itemize}
    \item Chiffrement des données (HTTPS/TLS).
    \item Protection contre les injections SQL et failles XSS.
    \item RGPD : Gestion du consentement pour les données personnelles.
\end{itemize}

\vfill
\begin{center}
    \textit{Document généré pour le projet sportif INSA 2026.}
\end{center}

\end{document}